%%%%%%%%%%%%%%%%%%%%%%%%%%%%%%%%%%%%%%%
% Deedy - One Page Two Column Resume
% LaTeX Template
% Version 1.2 (16/9/2014)
%
% Original author:
% Debarghya Das (http://debarghyadas.com)
%
% Original repository:
% https://github.com/deedydas/Deedy-Resume
%
% IMPORTANT: THIS TEMPLATE NEEDS TO BE COMPILED WITH XeLaTeX
%
% This template uses several fonts not included with Windows/Linux by
% default. If you get compilation errors saying a font is missing, find the line
% on which the font is used and either change it to a font included with your
% operating system or comment the line out to use the default font.
% 
%%%%%%%%%%%%%%%%%%%%%%%%%%%%%%%%%%%%%%
% 
% TODO:
% 1. Integrate biber/bibtex for article citation under publications.
% 2. Figure out a smoother way for the document to flow onto the next page.
% 3. Add styling information for a "Projects/Hacks" section.
% 4. Add location/address information
% 5. Merge OpenFont and MacFonts as a single sty with options.
% 
%%%%%%%%%%%%%%%%%%%%%%%%%%%%%%%%%%%%%%
%
% CHANGELOG:
% v1.1:
% 1. Fixed several compilation bugs with \renewcommand
% 2. Got Open-source fonts (Windows/Linux support)
% 3. Added Last Updated
% 4. Move Title styling into .sty
% 5. Commented .sty file.
%
%%%%%%%%%%%%%%%%%%%%%%%%%%%%%%%%%%%%%%%
%
% Known Issues:
% 1. Overflows onto second page if any column's contents are more than the
% vertical limit
% 2. Hacky space on the first bullet point on the second column.
%
%%%%%%%%%%%%%%%%%%%%%%%%%%%%%%%%%%%%%%


\documentclass[]{deedy-resume-openfont}
\usepackage{fancyhdr}
 
\pagestyle{fancy}
\fancyhf{}
 
\begin{document}

%%%%%%%%%%%%%%%%%%%%%%%%%%%%%%%%%%%%%%
%
%     LAST UPDATED DATE
%
%%%%%%%%%%%%%%%%%%%%%%%%%%%%%%%%%%%%%%
%\lastupdated

%%%%%%%%%%%%%%%%%%%%%%%%%%%%%%%%%%%%%%
%
%     TITLE NAME
%
%%%%%%%%%%%%%%%%%%%%%%%%%%%%%%%%%%%%%%
\namesection{Bryan}{Orabutt}{  
}

%%%%%%%%%%%%%%%%%%%%%%%%%%%%%%%%%%%%%%
%
%     COLUMN ONE
%
%%%%%%%%%%%%%%%%%%%%%%%%%%%%%%%%%%%%%%

\begin{minipage}[t]{0.31\textwidth} 
%%%%%%%%%%%%%%%%%%%%%%%%%%%%%%%%%%%%%%
%     Contact
%%%%%%%%%%%%%%%%%%%%%%%%%%%%%%%%%%%%%%

\section{Contact}
Phone: 217-299-8623\\
\urlstyle{same}\href{mailto:bryan@bryanorabutt.com}{bryan@bryanorabutt.com}\\
\urlstyle{same}\href{http://www.github.com/BryanOrabutt}{github.com/BryanOrabutt}\\
\urlstyle{same}\href{http://www.bryanorabutt.com}{bryanorabutt.com}\\
\urlstyle{same}\href{http://www.linkedin.com/in/borabutt}{LinkedIn: borabutt}
\sectionsep
%%%%%%%%%%%%%%%%%%%%%%%%%%%%%%%%%%%%%%
%     SKILLS
%%%%%%%%%%%%%%%%%%%%%%%%%%%%%%%%%%%%%%

\section{Skills}
\subsection{Programming}
C \textbullet{}   Python \textbullet{} MATLAB \textbullet{} C++ \\ 
OTPL \textbullet{} Verilog \textbullet{} VerilogA \\
Assembly \textbullet{} JSL
\sectionsep

\subsection{Tools}
Cadence IC Tools \textbullet{} KiCAD \\
Calibre DRC/LVS/PEX \textbullet{} Git\\
LTSpice \textbullet{} JMP \textbullet{} Visual Studio
\sectionsep

\subsection{Embedded Platforms}
Arduino \textbullet{} PSoC \textbullet{} Raspberry Pi\\
Beaglebone \textbullet{} ESP32
\sectionsep

%%%%%%%%%%%%%%%%%%%%%%%%%%%%%%%%%%%%%%
%     EDUCATION
%%%%%%%%%%%%%%%%%%%%%%%%%%%%%%%%%%%%%%

\section{Education} 
\subsection{Washington University \\in St. Louis} 
\vspace{0.1cm}
\descript{MS in Computer Engineering}
\location{May 2022| St. Louis, MO}
\sectionsep

\subsection{Southern Illinois \\University Edwardsville}
\vspace{0.1cm}
\descript{MS in Electrical Engineering}
\location{August 2018 | Edwardsville, IL}
\sectionsep
\descript{BS in Computer Engineering}
\location{December 2015 |  Edwardsville, IL}
\sectionsep

%%%%%%%%%%%%%%%%%%%%%%%%%%%%%%%%%%%%%%
%     Interests
%%%%%%%%%%%%%%%%%%%%%%%%%%%%%%%%%%%%%%

\section{Interests}
I am an engineer at heart and love to make things. I enjoy embedded programming, circuit design, and 3D printing.\\
\sectionsep
You can also often find me playing board games and table top RPGs with friends, or out foraging for wild mushrooms.



%%%%%%%%%%%%%%%%%%%%%%%%%%%%%%%%%%%%%%
%
%     COLUMN TWO
%
%%%%%%%%%%%%%%%%%%%%%%%%%%%%%%%%%%%%%%

\end{minipage} 
\hfill
\begin{minipage}[t]{0.66\textwidth} 

%%%%%%%%%%%%%%%%%%%%%%%%%%%%%%%%%%%%%%
%     EXPERIENCE
%%%%%%%%%%%%%%%%%%%%%%%%%%%%%%%%%%%%%%

\section{Experience}
\runsubsection{Intel}
\descript{| Product Development Engineer }
\location{Jan 2022 - Present | Hillsboro, OR}
\vspace{\topsep} % Hacky fix for awkward extra vertical space
\begin{tightemize}
\item Developed OTPL test programs for sorting singulated die. Collected and analyzed data for performing failure mode analysis and feeding back into the foundry for process improvements. 
\item Developed a python script to automate OTPL test plan generation for concurrent IP testing, reducing test time across multiple products.
\item Took charge of investigation of anomalous measurements resulting in new testing standards being implemented for specific product families.
\end{tightemize}
\sectionsep

%%%%%%%%%%%%%%%%%%%%%%%%%%%%%%%%%%%%%%
%     RESEARCH
%%%%%%%%%%%%%%%%%%%%%%%%%%%%%%%%%%%%%%

\section{Research}
\runsubsection{WUSTL Radiochemistry Lab}
\descript{| Doctoral Research}
\location{Jan 2019 – Jan 2022 | St. Louis, MO}
\begin{tightemize}
\item Worked with \textbf{Dr. Roger Chamberlain} and \textbf{Dr. Lee Sobotka} to develope Verilog-A models to simulate new mixed-mode pulse processing electronics.
\item Developed offline error correction algorithms for mixed-mode particle identifcation circuit.
\item Supported the design and simulation of new iterations of existing pulse processing ASICs for particle identification and energy measurement.
\item Designed software in C and MATLAB to facilitate test and characterization of packaged pulse processing ASICs prior to use in experiements.
\end{tightemize}
\sectionsep

\runsubsection{SIUE VLSI Design Lab}
\descript{| Master's Research}
\location{Jan 2016 – August 2018 | Edwardsville, IL}
\begin{tightemize}
\item Worked with \textbf{Dr. George Engel} to devlop a new analog ASIC for detecting the onset of radiation at a scintillator detector. 
\item Engaged in all aspects of the development: from design, schematic entry, simulation, layout, and final testing of the packaged silicon. 
\end{tightemize}
\sectionsep

%%%%%%%%%%%%%%%%%%%%%%%%%%%%%%%%%%%%%%
%     Projects
%%%%%%%%%%%%%%%%%%%%%%%%%%%%%%%%%%%%%%

\section{Academic Projects}
\runsubsection{ML Accelerator}
\descript{| Cadence Encounter, Verilog, C}
\location{Washington University in St. Louis}
\begin{tightemize}
\item Designed a machine learning accelerator for the MNIST data set comprised of multiple pipelined multiply accumulator units with an input/output RAM. 
\item Wrote a C program to model and train the neural network, obtaining the weight matrices for testing the accelerator design.
\end{tightemize}
\sectionsep

\runsubsection{Real-time GPIO Interrupt}
\descript{| Raspberry Pi, C, Linux, PSoC}
\location{Southern Illinois University Edwardsville}
\begin{tightemize}
\item Developed a project to reduce IRQ latency on the Raspberry Pi GPIO using a kernel driver and user space library. 
\item The user space library uses a character device to cause blocking reads until a GPIO interrupt is detected. 
\item The kernel driver uses a FIFO with real time scheduling to schedule IRQ handlers for each GPIO request, alerting user space via the character device.
\end{tightemize}
\sectionsep

%%%%%%%%%%%%%%%%%%%%%%%%%%%%%%%%%%%%%%
%     PUBLICATIONS
%%%%%%%%%%%%%%%%%%%%%%%%%%%%%%%%%%%%%%

\section{Publications} 
\renewcommand\refname{\vskip -1.5em} % Couldn't get this working from the .cls file
\bibliographystyle{abbrv}
\bibliography{publications}
\nocite{*}

\end{minipage} 
\end{document}  \documentclass[]{article}
